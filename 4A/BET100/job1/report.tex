\documentclass{article}
\usepackage[utf8]{inputenc}

\title{Job 1: Finding Important Problems}
\author{Karan Thukral (20460691)}
\date{28 September 2016}

\usepackage{natbib}
\usepackage{graphicx}
\linespread{2.0}

\begin{document}

\maketitle

\section{Secuity Leaks}
Recently Yahoo Inc. confirmed that the company was hacked in 2014 and it led to the leak of 500 million credentials [1]. Since the leak happened in 2014, this information has already been made available on the black market. Though the passwords are indeed encrypted and still secure, it is a possibility that the encryption will be broken soon and the passwords made available. The mail issue behind this relates to individuals reusing passwords on multiple websites and that makes any leaked email and password combos valuable. Due to this reuse getting access to a email and password combo can give the hacker access to numerous user accounts. Due to the rise of cloud computing and companies like Dropbox, users are storing more and more data online and protecting this data is extremely important. Beyond this issue some websites don't allow users to setup secure passwords and this list includes websites like Tangerine, BMO etc. There are technologies like two factor authentication but they are not used by all users or even all websites. There is a need to build new tool to make it extremely simple for users and developer to make their accounts and applications safer. Another aim of the tool would be to discourage users from reusing passwords and use the tool to come up with either unique passwords or permutations of existing passwords

\section{Heat Exhaustion}
As pointed in the article Toronto recently went through the worst heat wave of the summer [3] and existing research shows that heat exposure has been linked to increased aggression, violence, higher rates of suicides and head exhaustion [4][5]. Excessive heat exposure can have negative societal and mental impacts along with serious health issues like heat exhaustion and heat stroke. As per a research published in the journal Environmental Research [4], there are significant negative effects on emotional well-being when the temperature was above $21^\circ$ C. Aisde from emotional effects, an extimated 120 people per year in Toronto die from extream heat exposure [5]. There is already societal knowledge to keep hydrated in the summer but that is not always followed by individuals. One has to be actively looking at signs of heat exhaustion in people to be able to catch them. Due to this there is a need for something to be built that focuses on detecting early signs of heat exhaustion and recommending steps to improve it. The concerns related to excessive heat exposure are serious but at the same time easily preventable by simply limiting the exposure, increasing hydration and rest as long as the problem is caught early enough. 

\section{Mental Health}
According to research, approximately one third of Ontario adolescents are suffering from psychological distress or mental health issues [6]. Similarly, a lot of post-secondary students are suffering from mental health issues due to a rise in tuition and debt [7]. The first step in overcoming mental illness is talking about it but there is a lot of stigma present in society [8]. Due to this stigma, people suffering from mental health issues don't come forward and talk about it and instead suffer silently which can have negative effects regarding their recovery [8]. Due to the mental health epidemic there is a need for a way for people suffering to reach out and get help without being judged for it. Technology has the ability to help solve this problem giving patients an anonymous way to get information and ask questions. This can be done in several ways such as a forum, chat bot etc. Mental health is a very important issue due to the shear number of people suffering from it in Canada and rest of the world. As someone who has previously gone through depression, having a way to anonymously talk to something or someone would have been a huge help in fighting through it.

\section{Pay Discrimination}
Despite having made progress over the year there is still significant pay discrimination that can be observed in the market. Despite the same education black graduates in the US, still receive anywhere from \$5k to \$20k less than their white counterparts [9]. Similarly, women in the US can receive as little at 80\% of their male counterparts [10]. These numbers clearly show that there are clear gender and racial biases that still exist in society and have significant negative effects. An example of their racial biases are the racist statements made by Donald Trump as US presidential candidate. In order to allow all races and genders to prosper they need to have equal access to opportunity which is currently not the case. By using existing technology and the internet, one can build a platform for job seekers and employers that immediately remove any identifying information or any information that be used to judge the person's gender or race. This can be done during both the job application process and the job interview process to prevent decisions being affected from biases. Allowing everyone equal access to jobs and opportunities is an important cause and can be seen from the numbers provided.

\section{Body Cameras}
Over tthe last few years, there have been a number of shootings by cops involving unarmed black teens [11]. This can be attributed to racism, misuse of power or number of other things. These shootings have led to unrest in the US and the public calling for increased accountability from police officers and departments. "Black Lives Matter" is one of such examples. One way police departments are trying to solve this is by using body cameras [12]. Though there are numerous solutions available on the market, they are expensive and officers often don't like to wear them due to privacy concerns. In order to hold both the public and officers accountable, the technology needs to be accepted by both as a necessity. There is a need for a better and cheaper body cameras to exist that tries to find a balance in between public accountability concers and officer's privacy concerns. Making this technology more affordable and usable will allow more police departments to purchase them and make them a requirement for all officers.

\section{Online Harassment}
With the increasing popularity of the internet and social networking websites like Twitter there has been a sharp increase in harrasement on the internet [14]. One such example of this online harrasement was "gamergate" back in 2015 [15]. The excessive harrasement online especially against women has made the internet and unwelcoming place for individuals. Where the internet was created with net nutrality in mind where everyone and everything on the internt was created equal, the harassment has gone against this ideal. Where people can rely on the law to potentially protect them against harassment in the real world, the same cannot be said for the online world. Recently, a non-guilty verdict was passed on the first criminal trial for harassment on Twitter [16]. This was a big example of why we cannott trust the legal system to adress online harassment. Companies like Twitter are trying to improve their response agasint harassment but clearly hasn't done enough since people still recieve death and rape threats over the platform. There is a need for something to be created that can monitor the harassment happening online across not only a single website but the internet as a whole. Internet has provided us with access to virtually an endless amount of knowledge, we need to make sure everyone has access to it without having to worry about harassment. 

\section{Immigration}
With the recent Syrian immigration influx into Canada and the rest of the world there are numerous needs such in find cities for the families along with integrating them into the community [17]. Another huge concern for immigrants if finding jobs. Highly educated immigrants are unable to find jobs due to lack of "canadian" experience [18]. Immigrartion is not only an important issue due to the syrian crisis but also because Canada attracts a a large number of immigrants from all over the world [18]. With the large influx of immigrants there is an oppurtunity to build tools that can spread the incomming population across the regions based on factors such as job availability, industry presense, old age care, schools etc. The aim behind this would be to find the best location for the immigrants based on their needs. For an example, if a family moves with young kids, they would be recommended to stay in a city with good public or private schools etc. Immigrants are integral to the Canadian economy [19] and thus there is a need to invest into tools making their transition easier.

\section{Pedestrian Accidents}
Every 4 hours, a pedestrian is hit in Toronto and someone dies every 10 days [20]. This statistic is up 15\% over the past 5 years. About 163 pedestrians have been killed since 2011 [20]. By their mere nature, these deaths and accidents are something that is entirely predictable. On analyzing existing data, The Globe and Mail were able to find trends such as the victims are disproportionately over 65 and hit by a larger vehicle [20]. All this existing data from the Toronto Police Service and other existing police departments can be combined and used to gain more information on when and where most of these accidents happen. This data can be further used to plan new cities, roads, neighborhoods etc. According to statistics police estimate that drivers are at fault 50-60\% of times where as the pedestrians are at fault 25-30\% of time [20]. Due to the large number of incidents that happen every year and the fact that most of them (if not all) are preventable make this problem worth researching and solving.

\newpage
\section{References}
\noindent
[1] Bloomberg News.(2016, September 22). Yahoo says 'state-sponsored actor' hacked 500 million user accounts. Retrieved September 25, 2016, from http://www.theglobeandmail.com/technology/yahoo-set-to-confirm-massive-data-breach-recode/article31997906/\\
\noindent
[2] Braga, M. (2014, April 30). Why Canada's banks have weaker passwords than Twitter or Google. Retrieved September 25, 2016, from \\
http://www.theglobeandmail.com/technology/digital-culture/why-canadas-banks-have-weaker-passwords-than-twitter-or-google/article18325257/ \\
\noindent
[3] CECCO, L. (2016, August 10). Toronto braces for worst heat wave of the summer. Retrieved September 27, 2016, from \\
http://www.theglobeandmail.com/news/toronto/toronto-braces-for-worst-heat-wave-of-the-summer/article31338915/ \\
\noindent
[4] LEUNG, W. (2016, August 16). Studies show that hot weather brings out the worst in us. Retrieved September 25, 2016, from \\
http://www.theglobeandmail.com/life/health-and-fitness/health/studies-show-that-hot-weather-brings-out-the-worst-in-us/article31468759/\\
\noindent
[5] MCCOLL, K. (2014, April 30). Climate change and health: Extreme heat a 'silent' killer. Retrieved September 25, 2016, from http://www.theglobeandmail.com/life/health-and-fitness/health/climate-change-and-health-extreme-heat-a-silent-killer/article18343936/\\
\noindent
[6] BRAIT, E. (2016, July 21). One-third of Ontario adolescents report 'psychological distress', survey finds. Retrieved September 25, 2016, from \\
http://www.theglobeandmail.com/life/health-and-fitness/health/one-third-of-ontario-adolescents-report-psychological-distress-survey-finds/article31044218/\\
\noindent
[7] SAGAN, A. (2016, June 2). The mental health impact of rising debt: 'A lot of students suffer silently' Retrieved September 25, 2016, from \\
http://www.theglobeandmail.com/globe-investor/personal-finance/genymoney/post-secondary-schools-look-to-address-mental-health-impact-of-rising-student-debt/article30200722/\\
\noindent
[8] CLOSE, G., \& STUART, H. (2013, June 12). Overcoming mental illness means overcoming stigma. Retrieved September 25, 2016, from \\
http://www.theglobeandmail.com/opinion/overcoming-mental-illness-means-overcoming-stigma/article12480148/\\
\noindent
[9] Black Workers Still Make Less Than Whites With the Same Degree. (2016, September 15). Retrieved September 25, 2016, from https://www.bloomberg.com/features/2016-america-divided/salaries/\\
\noindent
[10] Kitroeff, N., \& Rodkin, J. (2015, October 20). The Real Payoff From an MBA Is Different for Men and Women. Retrieved September 25, 2016, from http://www.bloomberg.com/news/articles/2015-10-20/the-real-cost-of-an-mba-is-different-for-men-and-women\\
\noindent
[11] KORING, P. (2016, September 22). Police shootings of black men divide the United States of America. Retrieved September 25, 2016, from \\
http://www.theglobeandmail.com/news/world/police-shootings-of-black-men-divide-the-united-states-of-america/article32018691/\\
\noindent
[12] Weise, K. (2016, July 12). Will a Camera on Every Cop Make Everyone Safer? Taser Thinks So. Retrieved September 25, 2016, from \\
https://www.bloomberg.com/news/articles/2016-07-12/will-a-camera-on-every-cop-make-everyone-safer-taser-thinks-so\\
\noindent
[13] CASEY, L. (2016, April 4). Strapped police forces moving toward costly body cameras. Retrieved September 25, 2016, from \\
http://www.theglobeandmail.com/news/national/strapped-police-forces-moving-towards-costly-body-cameras/article29508602/\\
\noindent
[14] SCHICK, S. (2009, March 13). Web opens up new avenues for harassment. Retrieved September 25, 2016, from http://www.theglobeandmail.com/technology/web-opens-up-new-avenues-for-harassment/article680692/\\
\noindent
[15] ALANG, N. (2015, October 28). Where gamergate and violence are concerned, not all voices are equal. Retrieved September 25, 2016, from \\
http://www.theglobeandmail.com/opinion/where-gamergate-and-harassment-are-concerned-not-all-voices-are-equal/article27018003/\\
\noindent
[16] WOOLLEY, E. (2016, January 25). The Twitter trial: Vile harassment online is consequence-free. Retrieved September 25, 2016, from \\
http://www.theglobeandmail.com/opinion/the-twitter-trial-for-now-vitriolic-harassment-online-is-consequence-free/article28370250/\\
\noindent
[17] Brown, I. (2016, July 8). Finding a home, away from home: Refugees, sponsors and what it means to be Canadian. Retrieved September 25, 2016, from http://www.theglobeandmail.com/news/national/suleyman-family-finding-a-home-away-from-home/article30703243/\\
\noindent
[18] TRICHUR, R., \& GRANT, T. (2011, December 16). Shortchanging immigrants costs Canada. Retrieved September 25, 2016, from http://www.theglobeandmail.com/report-on-business/economy/jobs/shortchanging-immigrants-costs-canada/article4247734/\\
\noindent
[19] SAUL, J. R. (2016, July 22). Integral to Canada's economy, immigrants deserve more support. Retrieved September 25, 2016, from \\
http://www.theglobeandmail.com/news/national/integral-to-canadas-economy-immigrants-deserve-more-support/article31079752/\\
\noindent
[20] Moore, O. (2016, July 15). Fatal crossings: Where and how pedestrians die in Toronto. Retrieved September 25, 2016, from \\
http://www.theglobeandmail.com/news/toronto/more-than-160-pedestrians-killed-by-vehicles-since-2011/article30391640/
\end{document}
