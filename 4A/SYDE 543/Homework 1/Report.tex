\documentclass[12pt]{article}
\usepackage[utf8]{inputenc}

%THIS IS WHERE ALL THE SYDE STUFF IS
\usepackage{sydestyle}

% Other imports go here
\usepackage{graphicx}
\graphicspath{{figures/}}
\usepackage{amsmath}
\usepackage{hyperref}
\hypersetup{
    colorlinks,
    citecolor=black,
    filecolor=black,
    linkcolor=black,
    urlcolor=black
}


\title{Homework 1}
\classname{SYDE 543}
\author{Karan Thukral, 20460691, 4A}
\date{September 29, 2016}
\supervisortext{Course Instructor: Professor Shi Cao}

% Content here
\begin{document}

	% Start with a title page!
	\makereporttitle

	% Make a table of contents.
	\tableofcontents
	\newpage

	% Couldn't figure out how to automate this, but leave this line in to restart page numbers at 1 and with arabic numbers.
	\startarabicpagenumbers

	\section{Colour Blindness}
	The first example being examined is an application called "Trello". Trello is used in order to organize tasks across multiple people. This is done through the concepts of boards and heavily relies on colours to tag the tasks. Due to the heavy use of colour, the company has taken steps in order to improve their usability for colour blind users. There is an option available that can be toggled to launch the application in a colour blind more. A side by side comparison is shown in Figure \ref{trelloimg}.
	
	\begin{figure}[!ht]
		\centering
		\includegraphics[width=1.0\textwidth]{trelloImage}
		\caption{Colour Blind Mode (Left) vs Regular Mode (Right)}
		\label{trelloimg}
	\end{figure}
	
	As it can be seen in Figure \ref{trelloimg} there are subtle differences in how the colours are displayed. Figure \ref{trelloLabels} shows the difference in more detail.
	
	\begin{figure}[!ht]
		\centering
		\includegraphics[width=0.2\textwidth]{trelloLabels}
		\caption{Colour Blind Mode Label Colours}
		\label{trelloLabels}
	\end{figure}
	
	As seen in the two figures during colour blind mode, Trello doesn't only rely on differentiation through colours. Instead Trello uses shapes and textures within the colour to differentiate between them. As an example, the only solid colour used is blue, where as for purple Trello has diagonal lines through it. Similarly unique patterns are used to make the colours/labels differentiable. Trello is an excellent example of using the knowledge (visual senses) from the cognitive ergonomics to make the product usable by a larger population. Due to the high reliance of the application on colours, without the colour blind more it would have been extremely unusable for people suffering from colour blindness.
	
	\section{Sensory Adaption}
	This section will discuss the "Night Shift" feature on the iOS platform as an example for sensory adaption. Most displays act as an artificial source of lighting but also emit a significant amount of light in the spectrum of the blue wavelength. As per research conducted blue wavelengths are useful during the day because they help boost attention, reaction time and mood. The same wavelengths of light at night are disruptive and can negatively affect sleep [1]. In order to reduce or curb the negative effects of blue light, Apple added a new feature onto their iOS platform called "Night Shift". As the day continues, the phone screen changes the temperature of the light use and starts to use warmer tone after sundown. The aim behind this feature is to reduce the eye strain caused by a bright display especially at night (in darkness). The effect is shown in Figure \ref{nightshift}. This is a good example of the principle of adaption because it gives the user time to adjust to the change as its done gradually throughout the day. I believe that this is good design and utilizes existing research and cognitive ergonomics knowledge to provide the best user experience.
	
	\begin{figure}[!ht]
		\centering
		\includegraphics[width=0.5\textwidth]{nightshift}
		\caption{Night Shift On (Left) vs Off (Right) [2]}
		\label{nightshift}
	\end{figure}
	
	\section{Haptic Feedback and Attention}
	The next example used is the Apple Watch and particularly its map application. Due to the small screen of the watch, Apple had to use more than just the screen to convey information. As part of the map application Apple uses haptic feedback to alert the user of an upcoming turn. This is useful when walking or driving since it does not require the user to look at the watch while trying to navigate. If the only source of information was text on the watch, the user would have to constantly keep checking it which can be dangerous while walking as it diverts the users attention from their environment and traffic. 
	
	Despite being useful, I personally feel that it can be improved. Since the watch only alerts the user that a turn is approaching and not which turn to make, it requires the user to check the watch every time. This can lead to a lot of context switching for the user which diminishes the user experience. One way this can be improved it by allowing the user to customize the haptic feedback pattern. This is already a feature available for call notifications on the iPhone. For example, for a left turn the watch can vibrate once and for right it can vibrate twice. This can be done in different ways by relying on length or intensity of the vibration or number of vibrations etc. Having this ability would make it possible for the user to reach their destination without having to look at the watch every time it vibrates.
	
%	\section{Consistency}
%	In order to discuss the principle of consistency, I am going to use the design of the Shopify admin. In order to look at an example of the principle in the admin, lets focus on the design of the butttons. Shopify does a really good job of having a small set of buttton designs which used through the admin. Once such example is every distructive button like "Delete Product" is coloured red (to signify danger) and such an action is always followed by a popup dialog to confirm the action. Two different occourances of this are shown in Figure \ref{delete-product} and \ref{delete-discount}. Similarly the delete button is always the right button on the dialogue and the "Cancel" button has a white background. Any action that does not affect state of the store has a white background and blue text. 
%	
%	\begin{figure}[!ht]
%		\centering
%		\includegraphics[width=0.5\textwidth]{delete-product}
%		\caption{Delete Product Dialog}
%		\label{delete-product}
%	\end{figure}
%	
%	\begin{figure}[!ht]
%		\centering
%		\includegraphics[width=0.5\textwidth]{delete-discount}
%		\caption{Delete Discount Dialog}
%		\label{delete-discount}
%	\end{figure}
%	
%	The consistency throughout the admin reduces the congnivite workload for the user since the action type can be infered with only looking at the design and without having to read the text. One area where I feel like Shopify can improve is to use the red colour throughtout the admin. As an example, the "Delete Product" button the product page only becomes red when it is highlighted. Having it be a consistent red would inference extreamly easy.

\section{Adaption in Design}
In order to discuss another principle of adaption, the report uses "Adobe Lightroom" as an example. As per research in the field of cognitive ergonomics, it is known that people are better at telling differences than absolute values. This principle is used by Lightroom and other photo editing software. This section will be focusing on the design of the editing panel in Lightroom as shown in Figure \ref{lightroom}. 
	
	\begin{figure}[!ht]
		\centering
		\includegraphics[width=0.3\textwidth]{lightroom}
		\caption{Adobe Lightroom Editing Panel}
		\label{lightroom}
	\end{figure}
	
As it can be seen in Figure \ref{lightroom}, all sliders are placed at zero by default and the photographer can either increase it or decrease it. The application doesn't convey this change in absolute values but instead relies on using relative change ($\delta$) as compared to the original. This is a great example of a design that purposely removes any references to absolute values. This is good design since it only requires the user to judge a relative difference between their current state and final state. As an example, if the design instead used absolute values for exposure (candela), there is added burden on the user to have pre existing knowledge and experience with these concepts. Instead the user only has to decide to increase or decrease the exposure by some amount relative to its current state. 

\section{Gestalt Principles: Proximity}
Adobe Lightroom is used again in order to discuss the gestalt principles. This section focuses particularly on the use of proximity to encourage the user to perceive multiple options as a single group. Taking a look at Figure \ref{lightroom}, the designers have done a great job of using different amount of spacing in between the sliders to group them into sections. This idea is backed by the use of thin and thick lines to break up the sections and subsections. As an example, the exposure and contrast sliders are placed close to each other and within the same section due to their closely related nature and effect. Similarly highlights, shadows, whites and blacks are placed together and are also placed close to the exposure and contrast sliders. There is very deliberate break (line) between the mentioned sliders and the clarity slider. This emphasizes the fact that these sliders and tools belong to different groups. The placement of slides used is extremely useful due to the close proximity of related tools which are often used together in order to achieve a desired effect. Due to this, I believe that this is good example of using the gestalt principles in order to improve the user interface design.

\newpage
\section{References}
$[1]$ S. Hill, ''Is blue light keeping you up at night? we ask the experts.''\\ 
http://www.digitaltrends.com/mobile/does-blue-light-ruin-sleep-we-ask-an-expert/, July 26, 2015.\\
$[2]$ J. Reed, ''How to use night shift in ios 9.3.''\\
http://www.iphonefaq.org/archives/975151, Febuary 22, 2016.
	
	
\end{document}
